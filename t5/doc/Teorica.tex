\section{Theoretical Analysis}

\subsection{Description and Mathematical Considerations}

The Op-Amp (signal amplifier) used was considered ideal, which means that the internal impedance between v+ and v- is infinite and that there is no current flowing through it (v+ = v-). So, by connecting a capacitor ($C_1$) in series with the input voltage, we made a band pass filter, functioning as a high pass filter. Furthermore, in the final stage of the circuit, another capacitor ($C_2$) was connected in parallel with the output voltage, functioning as a low pass filter. Concluding, this circuit consists of a high pass filter, a signal amplifier and a low pass filter in series.

The circuit shown above was analyzed using Octave.

Initially we computed the input and the output impedance:

\begin{equation}
    Z_{in} = R_1 + \frac{1}{j*w*C}
\end{equation}

\begin{equation}
    Z_{out} = \frac{R_2}{j*w_0*C_2*(R_2+ \frac{1}{j w C_2})}
\end{equation}

\begin{table}[H]
\centering
\begin{tabularx}{0.6\textwidth} {
  | >{\raggedright\arraybackslash}X
  | >{\raggedleft\arraybackslash}X | }
 \hline
Input Impedance & 1.000000e+03 + -7.071068e+02j \\ \hline
Output Impedance & 6.666667e+02 + -4.714045e+02j\\ \hline

\end{tabularx}
\caption{Impedances}
\end{table}

Then we computed the gain:

\begin{equation}
    Gain = A_V * A_H * A_L
\end{equation}

\begin{equation}
A_{V}=  1+\frac{R3}{R4}
\end{equation}

\begin{equation}
A_{L}=  \frac{1}{1+R2*C2*s}
\end{equation}

\begin{equation}
A_{H}=  \frac{R1*C1*s}{1+R1*C1*s}
\end{equation}

Where $A_V$ is the gain that results from the OP-AMP (signal amplifier), $A_H$ and $A_L$ are the gains that correspond to the high pass and low pass filters, respectively.

Using the previous equations we obtained the gained presented:

\begin{table}[H]
\centering
\begin{tabularx}{0.6\textwidth} {
  | >{\raggedright\arraybackslash}X
  | >{\raggedleft\arraybackslash}X | }
 \hline
Gain & 1.006667e+02 \\ \hline
Gain & 4.005771e+01 dB \\ \hline

\end{tabularx}
\caption{Gain}
\end{table}

To better understand the frequency response of gain, we plot the following graph \ref{fig:gain}.  We can see that low and high frequencies have a a low gain and frequencies near to 1000 Hz have the maximum gain, as expected for a band pass filter.
In the first stage the high pass filter blocks the low frequencies and in the final stage we have a low pass filter that blocks high frequencies. To ensure an high gain, we must focus obtaining an high AV , because near the central frequency both AH and AL will be approximately 1. We can see as well in graph \ref{fig:freq} the plot for the theoretical phase of the output voltage which is similar to a normal bandpass filter.

\begin{figure}[H] \centering
\includegraphics[width=0.7\linewidth]{fase.eps}
\caption{Octave phase response}
\label{fig:freq}
\end{figure}

\begin{figure}[H] \centering
\includegraphics[width=0.7\linewidth]{teoria.eps}
\caption{Octave gain response}
\label{fig:gain}
\end{figure}
We also computed the low cutoff frequency and the high cut off frequency, and with those two we compute the central frequency using the following equation:

\begin{equation}
\omega_{L}= \frac{1}{R1*C1}
\end{equation}

\begin{equation}
\omega_{H}= \frac{1}{R2*C2}
\end{equation}

\begin{equation}
\omega_{0}= \sqrt{\omega_{L} * \omega_{H}}
\end{equation}

\begin{table}[H]
\centering
\begin{tabularx}{0.6\textwidth} {
  | >{\raggedright\arraybackslash}X
  | >{\raggedleft\arraybackslash}X | }
 \hline
Lower Cut Off Frequency & 4.545455e+03 rad/s\\ \hline
Higher Cut Off Frequency & 9.090909e+03 rad/s\\ \hline
Central Frequency & 6.428243e+03 rad/s\\ \hline

\end{tabularx}
\caption{Frequencies}
\end{table}

The figure of merit was also computed, following the expression presented in section 1.

\begin{table}[H]
\centering
\begin{tabularx}{0.6\textwidth} {
  | >{\raggedright\arraybackslash}X
  | >{\raggedleft\arraybackslash}X | }
 \hline
Central Frequency Deviation & 2.308672e+01 Hz \\ \hline
Gain Deviation & 5.771376e-02 dB \\ \hline
Cost & 1.428600e+04 MU \\ \hline
Merit & 3.024424e-06 \\ \hline

\end{tabularx}
\caption{Cost and merit}
\end{table}




