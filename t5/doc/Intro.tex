\section{Introduction}

For the fifth laboratory assignment on our Circuit Theory and Electronics Fundamentals course, we had to dimension and implement a Band Pass Filter (BPF) using an OP-AMP (Operational Amplifier) with a central frequency of 1KHz and a gain at central frequency of 40dB. An OP-AMP  is a DC-coupled high-gain electronic voltage amplifier with a differential input and, usually, a single-ended output. We had available a certain number of components to build the circuit that is shown in the image below. In order to study the behaviour of the circuit, we use Ngspice to measure the output voltage gain in the pass-band, the central frequency, and finally the input and output impedance at the central frequency. Then, we use Octave to do the theoretical analyses (Compute the frequency responses Vo(f)/Vi(f), the gain, and the input and output impedance at the central frequency. In this report, we will perform a comparison between our theoretical analysis and the simulation results, trying to explain any major discrepancies.
\par

To measure the quality of the filter the following equation was used:

\begin{equation}
    Merit=\frac{1}{cost*gaindeviation*centralfreqdeviation + 10^{-6}}
\end{equation}

\par

\begin{figure}[ht] \centering
\includegraphics[width=0.7\linewidth]{circ5.pdf}
\caption{Circuit}
\label{fig:circ}
\end{figure}
\par
A table with the values associated to each component is shown below. Note that the variables are in Volts, Ohms or Faradays.
\par
\begin{table}[H]
\centering
\begin{tabularx}{0.6\textwidth} {
  | >{\raggedright\arraybackslash}X
  | >{\raggedleft\arraybackslash}X | }
 \hline
R1 & 1.000000e+03 \\ \hline
R2 & 1.000000e+03 \\ \hline
R3 & 1.500000e+05 \\ \hline
R4 & 1.000000e+03 \\ \hline
C1 & 2.200000e-07 \\ \hline
C2 & 1.100000e-07 \\ \hline

\end{tabularx}
\caption{Circuit's components}
\end{table}
\clearpage
